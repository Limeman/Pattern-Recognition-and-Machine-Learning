\documentclass[]{article}
\usepackage{lmodern}
\usepackage{amssymb,amsmath}
\usepackage{ifxetex,ifluatex}
\usepackage{fixltx2e} % provides \textsubscript
\ifnum 0\ifxetex 1\fi\ifluatex 1\fi=0 % if pdftex
  \usepackage[T1]{fontenc}
  \usepackage[utf8]{inputenc}
\else % if luatex or xelatex
  \ifxetex
    \usepackage{mathspec}
  \else
    \usepackage{fontspec}
  \fi
  \defaultfontfeatures{Ligatures=TeX,Scale=MatchLowercase}
  \newcommand{\euro}{€}
\fi
% use upquote if available, for straight quotes in verbatim environments
\IfFileExists{upquote.sty}{\usepackage{upquote}}{}
% use microtype if available
\IfFileExists{microtype.sty}{%
\usepackage{microtype}
\UseMicrotypeSet[protrusion]{basicmath} % disable protrusion for tt fonts
}{}
\usepackage[margin=1in]{geometry}
\usepackage{hyperref}
\PassOptionsToPackage{usenames,dvipsnames}{color} % color is loaded by hyperref
\hypersetup{unicode=true,
            pdftitle={Assignment 3},
            pdfauthor={Per Emil Hammarlund, Albert Öst},
            pdfborder={0 0 0},
            breaklinks=true}
\urlstyle{same}  % don't use monospace font for urls
\usepackage{graphicx,grffile}
\makeatletter
\def\maxwidth{\ifdim\Gin@nat@width>\linewidth\linewidth\else\Gin@nat@width\fi}
\def\maxheight{\ifdim\Gin@nat@height>\textheight\textheight\else\Gin@nat@height\fi}
\makeatother
% Scale images if necessary, so that they will not overflow the page
% margins by default, and it is still possible to overwrite the defaults
% using explicit options in \includegraphics[width, height, ...]{}
\setkeys{Gin}{width=\maxwidth,height=\maxheight,keepaspectratio}
\setlength{\parindent}{0pt}
\setlength{\parskip}{6pt plus 2pt minus 1pt}
\setlength{\emergencystretch}{3em}  % prevent overfull lines
\providecommand{\tightlist}{%
  \setlength{\itemsep}{0pt}\setlength{\parskip}{0pt}}
\setcounter{secnumdepth}{0}

%%% Use protect on footnotes to avoid problems with footnotes in titles
\let\rmarkdownfootnote\footnote%
\def\footnote{\protect\rmarkdownfootnote}

%%% Change title format to be more compact
\usepackage{titling}

% Create subtitle command for use in maketitle
\newcommand{\subtitle}[1]{
  \posttitle{
    \begin{center}\large#1\end{center}
    }
}

\setlength{\droptitle}{-2em}

  \title{Assignment 3}
    \pretitle{\vspace{\droptitle}\centering\huge}
  \posttitle{\par}
    \author{Per Emil Hammarlund, Albert Öst}
    \preauthor{\centering\large\emph}
  \postauthor{\par}
      \predate{\centering\large\emph}
  \postdate{\par}
    \date{2019-05-05}


% Redefines (sub)paragraphs to behave more like sections
\ifx\paragraph\undefined\else
\let\oldparagraph\paragraph
\renewcommand{\paragraph}[1]{\oldparagraph{#1}\mbox{}}
\fi
\ifx\subparagraph\undefined\else
\let\oldsubparagraph\subparagraph
\renewcommand{\subparagraph}[1]{\oldsubparagraph{#1}\mbox{}}
\fi

\usepackage{float}
\let\origfigure\figure
\let\endorigfigure\endfigure
\renewenvironment{figure}[1][2] {
    \expandafter\origfigure\expandafter[H]
} {
    \endorigfigure
}

\begin{document}
\maketitle

\tableofcontents

\newpage

\section{Implementation of forward pass and log
prob}\label{implementation-of-forward-pass-and-log-prob}

\subsection{Forward pass}\label{forward-pass}

The forward pass was implemented using the following code:

\begin{verbatim}
function [alphaHat, c]=forward(mc,pX)
%--------------------------------------------------------
%Code Authors:
% Albert Öst
% Per Emil Hammarlund
%--------------------------------------------------------

% Get the number of observations
T = size(pX, 2);

% Get the number of hidden states in the markov chain
ns = nStates(mc);

% That means that the alhpas will be a T by ns matrix of probabilities
alphaHat = zeros(ns, T);

%-------------------- continue code from here, and delete error message

% Check if the markov chain is finite or not
isFinite = finiteDuration(mc);
if isFinite
    transProbs = mc.TransitionProb(:, 1: end - 1);
    c = zeros(T + 1, 1);
else
    transProbs = mc.TransitionProb;
    c = zeros(T, 1);
end

alphaHat(:,1) = mc.InitialProb .* pX(:,1);
c(1) = sum(alphaHat(:,1));

alphaHat(:,1) = alphaHat(:,1) ./ c(1);

for t=2:T
    alphaHat(:,t) = pX(:,t) .* (alphaHat(:,t - 1)' * transProbs)';
    c(t) = sum(alphaHat(:,t));
    alphaHat(:,t) = alphaHat(:,t) ./ c(t);
end

if isFinite
    c(T + 1) = sum(alphaHat(:,T) .* mc.TransitionProb(:,end));
end


end
\end{verbatim}

\newpage

\subsection{Implementation of log
prob}\label{implementation-of-log-prob}

The log prob was implemented using the following code:

\begin{verbatim}
%----------------------------------------------------
%Code Authors:
% Albert öst
% Per Emil Hammarlund
%----------------------------------------------------

function logP=logprob(hmm,x)
hmmSize=size(hmm);%size of hmm array
T=size(x,2);%number of vector samples in observed sequence
logP=zeros(hmmSize);%space for result
for i=1:numel(hmm)%for all HMM objects
    %Note: array elements can always be accessed as hmm(i),
    %regardless of hmmSize, even with multi-dimensional array.
    %
    %logP(i)= result for hmm(i)
    %continue coding from here, and delete the error message.
    [pX, logS] = prob(hmm(i).OutputDistr, x);

    % Scale the probs
    pX = (ones(size(pX, 1), 1) * exp(logS)) .* pX;

    [~, c] = hmm(i).StateGen.forward(pX);


    logP(i) = sum(log(c));

end;
\end{verbatim}

\newpage

\section{Verification of forward
pass}\label{verification-of-forward-pass}

To verify the forward pass, the model:

\[q = \begin{pmatrix}1 \\ 0 \end{pmatrix}\]
\[A = \begin{pmatrix}0.9 & 0.1 & 0 \\ 0 & 0.9 & 0.1 \end{pmatrix}\]

With the state conditional output:

\[g_1 = \mathcal{N}\left( 0, 1 \right)\]
\[g_2 = \mathcal{N}\left( 3, 2 \right)\]

Was constructed using the following code:

\begin{verbatim}
mc = MarkovChain([1; 0], [0.9 0.1 0; 0 0.9 0.1]);
g1 = GaussD('Mean', 0, 'StDev', 1);
g2 = GaussD('Mean', 3, 'StDev', 2);
\end{verbatim}

And \(pX\) was calculated by:

\begin{verbatim}
pX = prob([g1 g2], x);
\end{verbatim}

Now that the required parameters and model where complete, the function:

\begin{verbatim}
[alphaHat, c] = mc.forward(pX)
\end{verbatim}

Could now be tested, which gave the following output:

\begin{verbatim}
alphaHat =

   1.000000000000000   0.384704237490574   0.418874656659074
                   0   0.615295762509426   0.581125343340926

c =

          1.000000000000000
          0.162523466100529
          0.826580955035720
          0.058112534334093
\end{verbatim}

Which is the same values as what was desired in the lab instruction.

\newpage

\section{Validation of log prob}\label{validation-of-log-prob}

The same observations and model where used once again, but this time a
HMM was also constructed:

\begin{verbatim}
h = HMM(mc, [g1 g2]);
\end{verbatim}

The \textbf{logprob} function was then tested with:

\begin{verbatim}
logP = h.logprob(x)
\end{verbatim}

Which gave the following output:

\begin{verbatim}
logP =

  -9.187726979475208
\end{verbatim}

Which was the same value as what was desired in the lab instruction.

\end{document}
