\documentclass[]{article}
\usepackage{lmodern}
\usepackage{amssymb,amsmath}
\usepackage{ifxetex,ifluatex}
\usepackage{fixltx2e} % provides \textsubscript
\ifnum 0\ifxetex 1\fi\ifluatex 1\fi=0 % if pdftex
  \usepackage[T1]{fontenc}
  \usepackage[utf8]{inputenc}
\else % if luatex or xelatex
  \ifxetex
    \usepackage{mathspec}
  \else
    \usepackage{fontspec}
  \fi
  \defaultfontfeatures{Ligatures=TeX,Scale=MatchLowercase}
\fi
% use upquote if available, for straight quotes in verbatim environments
\IfFileExists{upquote.sty}{\usepackage{upquote}}{}
% use microtype if available
\IfFileExists{microtype.sty}{%
\usepackage{microtype}
\UseMicrotypeSet[protrusion]{basicmath} % disable protrusion for tt fonts
}{}
\usepackage[margin=1in]{geometry}
\usepackage{hyperref}
\hypersetup{unicode=true,
            pdftitle={Assignment 4},
            pdfauthor={Per Emil Hammarlund, Albert Öst},
            pdfborder={0 0 0},
            breaklinks=true}
\urlstyle{same}  % don't use monospace font for urls
\usepackage{graphicx,grffile}
\makeatletter
\def\maxwidth{\ifdim\Gin@nat@width>\linewidth\linewidth\else\Gin@nat@width\fi}
\def\maxheight{\ifdim\Gin@nat@height>\textheight\textheight\else\Gin@nat@height\fi}
\makeatother
% Scale images if necessary, so that they will not overflow the page
% margins by default, and it is still possible to overwrite the defaults
% using explicit options in \includegraphics[width, height, ...]{}
\setkeys{Gin}{width=\maxwidth,height=\maxheight,keepaspectratio}
\IfFileExists{parskip.sty}{%
\usepackage{parskip}
}{% else
\setlength{\parindent}{0pt}
\setlength{\parskip}{6pt plus 2pt minus 1pt}
}
\setlength{\emergencystretch}{3em}  % prevent overfull lines
\providecommand{\tightlist}{%
  \setlength{\itemsep}{0pt}\setlength{\parskip}{0pt}}
\setcounter{secnumdepth}{0}
% Redefines (sub)paragraphs to behave more like sections
\ifx\paragraph\undefined\else
\let\oldparagraph\paragraph
\renewcommand{\paragraph}[1]{\oldparagraph{#1}\mbox{}}
\fi
\ifx\subparagraph\undefined\else
\let\oldsubparagraph\subparagraph
\renewcommand{\subparagraph}[1]{\oldsubparagraph{#1}\mbox{}}
\fi

%%% Use protect on footnotes to avoid problems with footnotes in titles
\let\rmarkdownfootnote\footnote%
\def\footnote{\protect\rmarkdownfootnote}

%%% Change title format to be more compact
\usepackage{titling}

% Create subtitle command for use in maketitle
\providecommand{\subtitle}[1]{
  \posttitle{
    \begin{center}\large#1\end{center}
    }
}

\setlength{\droptitle}{-2em}

  \title{Assignment 4}
    \pretitle{\vspace{\droptitle}\centering\huge}
  \posttitle{\par}
    \author{Per Emil Hammarlund, Albert Öst}
    \preauthor{\centering\large\emph}
  \postauthor{\par}
      \predate{\centering\large\emph}
  \postdate{\par}
    \date{2019-05-10}


\begin{document}
\maketitle

\tableofcontents

\newpage

\hypertarget{implementation-of-backward-pass}{%
\section{Implementation of backward
pass}\label{implementation-of-backward-pass}}

The backward pass was implemented using the following code:

\begin{verbatim}
function betaHat=backward(mc,pX,c)
%--------------------------------------------------------
%Code Authors:
% Albert Öst
% Per Emil Hammarlund
%--------------------------------------------------------

%Initialization step
T=size(pX,2);%Number of observations

betaHat = zeros(size(pX));

% Depending on if the HMM is finite or not, we get different computations
isFinite = finiteDuration(mc);
if isFinite
    betaHat(:, end) = mc.TransitionProb(:, end)./(c(end-1)*c(end));
    A = mc.TransitionProb(:, 1:end - 1);
else
    betaHat(:, end) = ones(size(pX, 1), 1)./c(end);
    A = mc.TransitionProb;
end


% Backward step
for t = T-1:-1:1
    betaHat(:, t) = A * (pX(:, t + 1) .* betaHat(:, t + 1))  ./ c(t);
end

end
\end{verbatim}

\newpage

\hypertarget{validation-of-backward-pass}{%
\section{Validation of backward
pass}\label{validation-of-backward-pass}}

The validation of the backward pass was implemented using the following
code:

\begin{verbatim}
format long
clear

% Observations
x = [-0.2 2.6 1.3];

% Infinite
%mc = MarkovChain([0.75; 0.25], [0.99 0.01; 0.03 0.97]);
% Finite
mc = MarkovChain([1; 0], [0.9 0.1 0; 0 0.9 0.1]);
g1 = GaussD('Mean', 0, 'StDev', 1);
g2 = GaussD('Mean', 3, 'StDev', 2);

pX = prob([g1, g2], x);

[~, c] = mc.forward(pX);

betaHat = mc.backward(pX, c)
\end{verbatim}

\hypertarget{result-from-the-test-run}{%
\subsection{Result from the test run}\label{result-from-the-test-run}}

The test run gave the following result:

\begin{verbatim}
betaHat =

   1.000000000000000   1.038935709330079                   0
   8.415379245573641   9.350421383970712   2.081827732555444
\end{verbatim}

Which was very close to the desired values in the lab instruction. Here
\(c\) was calculated using the value returned from the forward
algorithm. In the lab instruction \(c\) was rounded to 4 decimal points
in each element. \(c\) had the following values from the forward pass:

\begin{verbatim}
c =

   1.000000000000000
   0.162523466100529
   0.826580955035720
   0.058112534334093
\end{verbatim}

These values where rounded to 4 decimal points, and \(c\) was assignted
them:

\begin{verbatim}
c = [1.0000, 0.1625, 0.8266, 0.0581];
\end{verbatim}

And the same code was used again, this gave the following result:

\begin{verbatim}
betaHat =

   1.000337126754333   1.039285962353727                   0
   8.418216295065184   9.353573661183537   2.082228884429218
\end{verbatim}

Which are the exactly the values that where desired in the lab
instruction!


\end{document}
